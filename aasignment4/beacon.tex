\documentclass[a4paper,12pt]{article}
\begin{document}


\begin{Huge}
\begin{center}
\begin{normalsize}

\textbf{MAKERERE UNIVERSITY } \\
\textbf{FACULTY OF COMPUTING AND INFORMATICS TECHNOLOGY} \\
\textbf{DEPARTMENT OF COMPUTER SCIENCE} \\
\textbf{BACHELOR OF SCIENCE IN COMPUTER SCIENCE} \\
\textbf{BIT 2207 RESEARCH METHODOLOGY} \\
\textbf{YEAR 2} \\


\textbf{\sc MASIGA DAVID KELVIN } \\
\textbf{\sc Reg No: 16/U/579 } \\
\textbf{\sc std No: 216000507}\\
\end{normalsize}
\end{center}
\end{Huge}
\newpage

\title{LITERATURE REVIEW ON BEACONS.}
\maketitle    
\section{Introduction:}
Beacons are small Bluetooth radio transmitter device designed to attract attention to a specific location. They repeatedly transmit a single signal that other devices can see. Beacons don't have any data on them. They don't usually connect to the Internet. They are a very simple device. They have a universally unique identifier, a major and minor.\cite{Patrick} 
\section{background:}                                                                                                                                                                                    
\paragraph{\sl Apple's iBeacon was the first standardized BLE beacon platform. The company introduced the technology during its summer 2013 Worldwide Developers Conference, when it added iBeacon support to iOS 7. The iBeacon platform lets developers build mobile apps that can receive transmissions, such as location-aware notifications, from iBeacon-compatible devices.}
\paragraph{\sl On Dec. 6, 2013, Apple installed iBeacons in all of its 254 U.S. retail stores. Shoppers with the Apple Store app installed on their Bluetooth-enabled, iOS devices with active location services can receive in-store notifications about deals, new products and more.}
\section{overview}
In today's noisy culture, it's increasingly difficult for enterprise marketers to reach customers and prospects, much less prompt them to take desired actions. However, this complex challenge appears to have at least one relatively simple solution: Bluetooth Low Energy (BLE) "beacons." Also known as "proximity beacons," the inexpensive devices transmit relevant, targeted messages and information to nearby mobile devices.\cite{James} 
\paragraph{\sl In-store retail and offline payments are in the first wave of beacon applications. Retail outlets are adopting beacons to provide customers with product information, flash sales or deals, and to speed up the checkout process with a completely contactless payments system}.\cite{Tony} 
 
 \begin{thebibliography}{9}
 \bibitem{Patrick} Patrick Leddy. \textit{"10 Things About Bluetooth Beacons You Need to Know"},
 Internet:http://academy.pulsatehq.com/bluetooth-beacons, Jul. 15, 2015, [Mar. 9, 2018].
\bibitem{James} James A.Martin. \textit{"6 things marketers need to know about beacons"}, 
Internet:https://www.cio.com/article/3037354/marketing/6-things-marketers-need-to-know-about-beacons.html, Feb. 24, 2016 [Mar. 9, 2018].
\bibitem{Tony} Tony Danova. \textit{"What They Are, How They Work, And Why Apple's iBeacon Technology Is Ahead of The Pack"} 
Internet: http://www.businessinsider.com/beacons-and-ibeacons-create-a-new-market-2013-12?IR=T, Oct. 23, 2014, [Mar. 9, 2018].

\end{thebibliography}

\end{document}